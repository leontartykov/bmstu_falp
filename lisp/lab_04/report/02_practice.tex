\chapter{Практические задания}
\section{}
\begin{lstlisting}
	(setf lst1 '(a b))
	(setf lst2 '(c d))
	
	(cons lst1 lst2)   ; ((a b) c d) 
	(list lst1 lst2)   ; ((a b) (c d))
	(append lst1 lst2) ; (a b c d)
\end{lstlisting}
\section{Каковы результаты вычисления следующих выражений?}
\begin{lstlisting}
	(reverse ()) 			; nil
	(last ())				; nil
	(reverse '(a))			; (a)
	(last '(a))				; (a)
	(reverse '((a b c)))	; ((a b c))
	(last '((a b c)))		; ((a b c))
\end{lstlisting}

\section{Написать, по крайней мере, два варианта функции, которая возвращает последний элемент своего списка-аргумента.}
\begin{lstlisting}
	(defun last_elem_1 (arg) (car (last arg)))
	
	(defun last_elem_2 (arg) (first (reverse arg)))
\end{lstlisting}

\section{Написать, по крайней мере, два варианта функции, которая возвращает свой список-аргумент без последнего элемента.}
\begin{lstlisting}
	(defun without_last_1 (arg) (reverse (cdr (reverse arg))))
	
	
\end{lstlisting}

\section{Игра в кости}
Простой вариант игры в кости, в котором бросаются две правильные кости. Если
сумма выпавших очков равна 7 или 11 -- выигрыш, если выпало (1,1) или (6,6) --- игрок право снова бросить кости, во всех остальных случаях ход переходит ко второму игроку, но запоминается сумма выпавших очков. Если второй игрок не выигрывает абсолютно, то выигрывает тот игрок, у которого больше очков. Результат игры и значения выпавших костей выводить на экран с помощью функции print.

