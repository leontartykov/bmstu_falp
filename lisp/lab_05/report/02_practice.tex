\chapter{Практические задания}
\section{Написать функцию, которая по своему списку-аргументу lst определяет, является ли он палиндромом.}
\begin{lstlisting}[basicstyle=\footnotesize, caption=Задание 1]
	(defun is_palindrom (lst) 
		(equal lst (reverse lst)))
\end{lstlisting}

\section{Написать предикат set-equal, который возвращает t, если два его множества-аргумента содержат одни и те же элементы, порядок которых не имеет значения.}
\begin{lstlisting}[basicstyle=\footnotesize, caption=Задание 2]
(defun is_elem_in_set (elem set)
	(cond ((null set) nil)
		((= elem (car set)) T)
		(T (is_elem_in_set elem (cdr set)))
	)
)

(defun is_equal_set (set_1 set_2)
	(cond ((null set_1))
		  ((is_elem_in_set (car set_1) set_2) 
		   (is_equal_set (cdr set_1) set_2)
		  )
	   	  (T nil)
	)
)



(defun set_equal (set_1 set_2)
	(if (= (length set_1) (length set_2)) 
	    (is_equal_set set_1 set_2) 
	    'Ошибка
	)
)
\end{lstlisting}

\section{Напишите свои необходимые функции, которые обрабатывают таблицу из 4-х точечных пар: (страна . столица), и возвращают по стране - столицу, а по столице — страну.}
\begin{lstlisting}[basicstyle=\footnotesize, caption=Задание 3]
(defvar table)
(setq table '((Russia . Moscow) (Ireland . Dublin) (Japan . Tokyo) 
              (China . Pekin)))

(defun get_country (capital)
	(let ((country 0))
		 (cond ((mapcar #'(lambda (x) (cond ((eq (cdr x) capital) 
									 (setq country (car x))))) 
									 table) 
				country)
		 )
	)
)

(defun get_capital (country)
	(let ((capital 0))
		 (cond ((mapcar #'(lambda (x) (cond ((eq (car x) country) 
		                             (setq capital (cdr x))))) 
		                             table) 
		        capital)
		 )
	)
)
\end{lstlisting}

\section{Напишите функцию swap-first-last, которая переставляет в списке-аргументе первый и последний элементы.}
\begin{lstlisting}[basicstyle=\footnotesize, caption=Задание 4]
(defun swap_first_last (lst)
	(cons (car (last lst)) 
	      (nconc (cdr (nreverse (cdr (nreverse lst)))) (cons (car lst) nil)))
)
\end{lstlisting}

\section{Напишите функцию swap-two-element, которая переставляет в списке- аргументе два указанных своими порядковыми номерами элемента в этом списке.}
\begin{lstlisting}[basicstyle=\footnotesize, caption=Задание 5]
(defvar first_elem)
(defvar second_elem)

(defun append_elem (lst index index_1 index_2)
	(cond ((null lst) nil)
		((= index index_1) 
		 (cons second_elem (append_elem (cdr lst) (+ index 1) index_1 index_2)))
		((= index index_2) 
		 (cons first_elem (append_elem (cdr lst) (+ index 1) index_1 index_2)))
		(T (cons (car lst) 
		   (append_elem (cdr lst) (+ index 1) index_1 index_2)))
	)
)
\end{lstlisting}
\newpage
\begin{lstlisting}[basicstyle=\footnotesize, caption=Задание 5 (продолжение)]
(defun swap_two_elements (lst index_1 index_2)
	(setq first_elem (nth index_1 lst))
	(setq second_elem (nth index_2 lst))
	(let ((len_lst (length lst))
		  (index 0))
		 (cond ((or (< index_1 0) (> index_1 len_lst)) "Некорректный размер")
	            ((or (< index_2 0) (> index_2 len_lst)) "Некорректный размер")
	            ((>= index_1 index_2) "Некорректный размер")
	            (T (append_elem lst 0 index_1 index_2))
	     )
	)
)
\end{lstlisting}

\section{Напишите две функции, swap-to-left и swap-to-right, которые производят одну круговую перестановку в списке-аргументе влево и вправо, соответственно.}
\begin{lstlisting}[basicstyle=\footnotesize, caption=Задание 6]
(defun without_last (lst)
	(nreverse (cdr (nreverse lst)))
)

(defun swap_to_right (lst)
	(cons (car (last lst)) (without_last lst))
)

(defun swap_to_left (lst)
	(nconc (cdr lst) (cons (car lst) nil))
)
\end{lstlisting}

\section{Напишите функцию, которая добавляет к множеству двухэлементных списков новый двухэлементный список, если его там нет.}
\begin{lstlisting}[basicstyle=\footnotesize, caption=Задание 7]
(defun check_lists_in_set (set_lists)
	(cond ((null set_lists) T)
		((not (listp (car set_lists))) nil)
		((if (= 2 (length (car set_lists))) 
		     (check_lists_in_set (cdr set_lists)) nil))
	)
)

(defun is_in_set (new_list set_lists)
	(cond ((null set_lists) T)
		  ((not (equal new_list (car set_lists))) 
		   (is_in_set new_list (cdr set_lists)))
	)
)

(defun append_new_elem (new_list set_lists)
	(cond ((not (listp new_list)) "Ошибка")
		  ((not (listp set_lists)) "Ошибка")
 		  ((not (check_lists_in_set set_lists)) "Ошибка")
		  ((is_in_set new_list set_lists) (nconc set_lists `(,new_list)))
		  (T set_lists)
	)
)
\end{lstlisting}

\section{Напишите функцию, которая умножает на заданное число-аргумент первый числовой элемент списка из заданного 3-х элементного списка- аргумента, когда
	a) все элементы списка --- числа,
	б) элементы списка -- любые объекты.}
\begin{lstlisting}[basicstyle=\footnotesize, caption=Задание 8]
(defun multiply_var_1 (lst number)
	(cond ((null lst) nil)
		   (T (cons (* number (car lst)) (cdr lst)))
	)
)

(defun multiply_var_2 (lst number)
	(cond ((null lst) nil)
		  (T (if (numberp (car lst))
				 (cons (* number (car lst)) (cdr lst)) 
				 (cons (car lst) (multiply_var_2 (cdr lst) number)))
		  )
	)
)

(defun check_lst (lst number)
	(cond ((not (numberp number)) nil)
		  ((not (listp lst)) nil)
		  ((not (= 3 (length lst))) nil)
		  (T T)
	)
)

(defun multiply_list_by_number_1 (lst number)
	(cond ((check_lst lst number) (multiply_var_1 lst number)))
)
(defun multiply_list_by_number_2 (lst number)
	(if (check_lst lst number) (multiply_var_2 lst number) "Ошибка")
)
\end{lstlisting}

\section{Напишите функцию, select-between, которая из списка-аргумента из 5 чисел выбирает только те, которые расположены между двумя указанными границами-аргументами и возвращает их в виде списка (упорядоченного по возрастанию списка чисел).}
\newpage
\begin{lstlisting}[basicstyle=\footnotesize, caption=Задание 9]
(defun select_between (lst board_left board_right)
	(let ((result_lst nil))
		(setq result_lst (if (check_input_data lst board_left board_right) 
		(find_left_board lst board_left board_right 0) 
		nil
						 )
		)
		(if (not (null result_lst)) (bubble_sort_asc result_lst) "Ошибка")
	)
)

(defun check_input_data (lst board_left board_right)
	(cond ((not (= (length lst) 5)) nil)
		  ((< board_left 0) nil)
		  ((>= board_right 5) nil)
		  ((<= board_right board_left) nil)
		  (T T)
	)
)

(defun find_left_board (lst board_left board_right index_board)
	(if (= index_board board_left) 
	    (find_right_board (cdr lst) board_right (+ index_board 1))
		(find_left_board (cdr lst) board_left board_right (+ index_board 1))   
	)
)



(defun find_right_board (lst board_right index_board)
	(if (= index_board board_right)
		(cons (car lst) nil) 
		(cons (car lst) (find_right_board (cdr lst) board_right (+ index_board 1)))
	)
)

(defun bubble (lst)
	(cond ((atom (cdr lst)) lst)
		  ((> (car lst) (cadr lst)) 
			(cons (cadr lst) (bubble (cons (car lst) (cddr lst)) ))
		  )
		  (T lst)
	)
)


(defun bubble_sort_asc (lst)
	(cond ((atom (cdr lst)) lst)
		   (T (bubble (cons (car lst) (bubble_sort_asc (cdr lst)))))
	)
)
\end{lstlisting}