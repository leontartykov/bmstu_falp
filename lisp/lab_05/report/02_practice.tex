\chapter{Практические задания}
\section{Написать функцию, которая по своему списку-аргументу lst определяет, является ли он палиндромом.}
\begin{lstlisting}
	(defun is_palindrom (lst) 
		(equal lst (reverse lst)))
\end{lstlisting}

\section{Написать предикат set-equal, который возвращает t, если два его множества-аргумента содержат одни и те же элементы, порядок которых не имеет значения.}
;lkfe
\section{Напишите свои необходимые функции, которые обрабатывают таблицу из 4-х точечных пар: (страна . столица), и возвращают по стране - столицу, а по столице — страну.}
ke
\section{Напишите функцию swap-first-last, которая переставляет в списке-аргументе первый и последний элементы.}
\section{Напишите функцию swap-two-element, которая переставляет в списке- аргументе два указанных своими порядковыми номерами элемента в этом списке.}
\section{Напишите две функции, swap-to-left и swap-to-right, которые производят одну круговую перестановку в списке-аргументе влево и вправо, соответственно}.
\section{Напишите функцию, которая добавляет к множеству двухэлементных списков новый двухэлементный список, если его там нет.}
\section{Напишите функцию, которая умножает на заданное число-аргумент первый числовой элемент списка из заданного 3-х элементного списка-аргумента, когда
	a) все элементы списка --- числа,
	6) элементы списка -- любые объекты.}
\section{Напишите функцию, select-between, которая из списка-аргумента из 5 чисел выбирает только те, которые расположены между двумя указанными границами-аргументами и возвращает их в виде списка (упорядоченного по возрастанию списка чисел).}