\chapter{Теоретические вопросы}

\section{Базис Lisp}
\textit{Базис} - минимальный набор конструкций языка и структур данных, с  помощью которых можно решить любую задачу. 

Базис Lisp образуют: атомы, структуры, базовые функции, базовые функционалы.

\textit{Базисные функции} -- минимальный набор функций, которые позволяют решить любую задачу.

\section{Классификация функций}
\begin{enumerate}
	\item Чистые математические функции.
	
	Имеют фиксированное число аргументов и возвращают один результат. Сначала вычисляются все аргументы, затем к ним применяется исходная функция.
	
	\item Специальные функции (формы).
	
	\textit{Специальные функции} -- функции, у которых переменное число аргументов или они обрабатываются по-разному (один вычисляется, другой - нет).
	
	\item Псевдофункции.
	
	\textit{Псевдофункции} -- функции, которые создают "спецэффекты"; например, вывод на экран.
	
	\item Функции с вариантами значений -- выбирают какое-то одно значение.
	
	\item Функционалы.
	
	\textit{Функционалы} -- функции, которые в качестве аргументов используют функции или возвращают в качестве результата функцию. Также они называются \textit{функциями более высокого порядка}. Позволяют создавать \textit{синтаксически управляемые программы} (программы, которые сами создают какие-то функции; эти функции затем выполняются). 
	
	\item Рекурсивные.
\end{enumerate}

\section{Способы создания функций}

\begin{enumerate}
	\item lambda-выражение. Данный способ представлен с помощью формулы (\ref{eq:lambda}).
	
	\begin{equation}
		\label{eq:lambda}
		(lambda \ \lambda \text{-} \textup{список} \ \textup{форма}),
	\end{equation}

	где $\lambda$-$\textup{список}$ -- список формальных параметров, $\textup{форма}$ -- тело функции.
	
	lambda$\text{-} \textup{выражение}$ не хранится в памяти и не имеет имени. Вычисляется сразу же. Используется для повторных вычислений.
	
	Вызов lambda$\text{-} \textup{функции}$ выполняется по формуле (\ref{eq:lambda_function}).
	
	\begin{equation}
		\label{eq:lambda_function}
		(\lambda \text{-} \textup{выражение} \ \textup{последовательность форм})
	\end{equation}
	
	\item С помощью \textit{defun} по формуле (\ref{eq:defun}). 
	
	\begin{equation}
		\label{eq:defun}
		(defun \ f \ \lambda \text{-} \textup{выражение})
	\end{equation}

	Система по имени символьного атома находит его определение.
\end{enumerate}

\section{Работа функциями cond, if, and/or}
\begin{enumerate}
	
\item Функция \textbf{cond} -- средство разветвления вычислений.
Вызов функции cond представлен по формуле (\ref{eq:cond}).

\begin{equation}
	\label{eq:cond}
	(cond\ (p_1\ e_1)\ (p_2\ e_2)\ ...\ (p_n\ e_n)),\ n \geq 1
\end{equation}

Обращение к функции cond называется \textbf{условным выражением}. Выражения $(p_i, e_i)$ -- \textbf{ветви} условного выражения; выражения-формы $p_i$ -- \textbf{условия} ветвей.

Порядок вычисления условного выражения:
\begin{enumerate}
	\item Последовательно вычисляются условия ветвей до тех пор, пока не встретится выражение-форма $p_i$, значение которой отлично от Nil.
	\item Вычисляются выражения $e_i$ соответствующей ветви и его значение возвращается в качестве значения функции cond.
	\item Если все условия $p_i$ имеют значение Nil, то значением условного выражения становится Nil.
\end{enumerate}

Функция \textbf{if}. Вызов функции if представлен по формуле (\ref{eq:if}).

 \begin{equation}
 	\label{eq:if}
 	(if\ (test\text{-}clause)\ (action\_t)\ (action\_f))
 \end{equation}

В случае, если условие test истинно, то вычисляется $action\_t$, иначе -- $action\_f$.

\item Логическая функция \textbf{and}. 

Вызов функции and представлен по формуле (\ref{eq:and}).
\begin{equation}
	\label{eq:and}
	(and\ e_1\ e_2\ ...\ e_n), n \geq 0
\end{equation}

При работе функции and последовательно слева-направо вычисляются аргументы функции $e_i$  -- до тех пор, пока не встретится значение, равное nil -- вычисление прерывается и значение функции равно nil. Если же были вычислены все значения $e_i$ и оказалось, что все они отличны от nil, то результирующим значением функции and будет значение последнего выражения $e_n$.

При $n=0$ значение функции and равно T. Значением функции and может быть не только T и nil, но и произвольный атом или списк.

\item Логическая функция \textbf{or}.

Вызов функции and представлен по формуле (\ref{eq:or}).
\begin{equation}
	\label{eq:or}
	(or\ e_1\ e_2\ ...\ e_n), n \geq 0
\end{equation}

При выполнении вызова последовательно вычисляются аргументы $e_i$
(слева-направо) -- до тех пор, пока не встретится значение $e_i$, отличное от nil. В этом случае вычисление прерывается и значение функции равно значению $e_i$. Если же вычислены значения всех аргументов и они равны nil, то результирующее значение функции
равно nil.

При $n=0$ значение функции or равно nil. Значением функции or может быть не только T и nil, но и произвольный атом или список.

\end{enumerate}