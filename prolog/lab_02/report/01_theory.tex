\chapter{Теоретическая часть}
\section{Элементы языка Prolog}
\textbf{Элементом языка Prolog} является \textit{терм}. \textbf{Терм} -- это один из следующих вариантов:
\begin{enumerate}
	\item \textit{константа}.
		\begin{itemize}
			\item Символьный атом. Обозначает объект в предметной области. Начинается со \textit{строчной} буквы.
			\item Числовая константа.
			\item Строка (в кавычках).
		\end{itemize}
	\item \textit{Переменная}. Необходима для компактности и повышения уровня абстракции.
		\begin{itemize}
			\item Именованная. Служит для передачи значения из базы знаний. Начинается с заглавной буквы. Уникальна в рамках одного предложения.
			\item Анонимная. Не может быть связана с каким-то значением. Начинается с символа "\_" (без кавычек). Уникальна всегда.
		\end{itemize}
	\item \textit{Составной терм}. Позволяет зафиксировать структуру конкретного объекта. Синтаксическая форма составного терма приведена в формуле (\ref{eq:complex_term}):
	\begin{equation}
		\label{eq:complex_term}
		f(t_{1},\ t_{2},...,t_{n}),
	\end{equation}
	где \textit{f} является функциональным символом (главный функтор) -- задает имя отношения, а $t_{1},\ t_{2},...,t_{n}$ -- последовательность термов. 
\end{enumerate} 

\section{Программа на языке Prolog}
\textbf{Программа на Прологе} состоит из базы знаний и вопроса. База знаний в свою очередь состоит из \textit{фактов} (аксиом) и \textit{правил} (теорем), причем они должны быть истинностными.

\textbf{Структура программы} состоит из следующих разделов (не все из них должны присутствовать):
\begin{itemize}
	\item директивы компилятора -- зарезервированные символьные константы;
	\item CONSTANTS -- раздел описания констант;
	\item DOMAINS -- раздел описания доменов;
	\item DATABASE -- раздел описания предикатов внутренней базы знаний;
	\item PREDICATES -- раздел описания предикатов;
	\item CLAUSES -- раздел описания предложений базы знаний;
	\item GOAL -- раздел описания вопроса.
\end{itemize}

\textbf{Факт} является частным случаем правила (имеет заголовок, но не имеет тела).

Факты и правила могут быть:
\begin{spacing}{0.7}
	\begin{itemize}
		\item[-] основными -- не содержат переменные;
		\item[-] неосновными -- содержат переменные. 
	\end{itemize}
\end{spacing}

\textbf{Правило} является условной истиной. Синтаксическая форма правила представлена в формуле (\ref{eq:rule}):
\begin{equation}
	\label{eq:rule}
	A\ :-\ B_{1}, B_{2},...,B_{N},
\end{equation}
где $A$ -- \textit{заголовок} (является составным термом), $B_{1}, B_{2},...,B_{N}$ -- \textit{тело} (являются составными термами). Символ ":-"\ обозначает "если".

В заголовке формулируются знания о том, что между аргументами тела есть связь. Тело содержит условие истинности.

\textbf{Вопрос} используется для определения выполнения отношения между описанными в программе объектами. Относительно правила является его частным случаем (состоит только из тела).

\section{Реализация программы на Prolog}
\textbf{Целью системы} (Prolog) является ответ на вопрос: либо "да", либо "нет". Таким образом, ей необходимо подобрать знания -- сравнить вопрос и заголовок факта или правила по формальному признаку -- с помощью механизма унификации. Данный механизм встроен в систему и недоступен пользователю.

\section{Подстановки и примеры терма}
\textbf{Подстановкой} называется множество пар вида $\{x_{i}\ =\ t_{i}\}$, где ${x_{i}}$ -- переменная, а $t_{i}$ -- основной терм (т.е. значение для $x_{i}$). Обозначается символом $\theta$.

\textbf{Пример} терма А -- терм B, если существует такая подстановка $\theta$, что $B=A\theta$, где $A\theta$ -- результат подстановки.

\textbf{Общий пример} термов A и B -- терм С, если существуют такие постановки $\theta_{1}$ и $\theta_{2}$, что $C = A\theta_{1}$ и $C = B\theta_{2}$ ($\theta_{1}$ $\theta_{2}$ могут совпадать).