\chapter{Контрольные вопросы}
\begin{enumerate}
	\item \textbf{В каком случае система запускает алгоритм унификации? (Как эту необходимость на формальном уровне распознает система?)} -- Алгоритм унификации запускается каждый раз при сравнивании двух термов, чтобы ответить на вопрос. На формальном уровне распознает по заголовкам.
	\item \textbf{Каковы назначение и результат использования алгоритма унификации?} -- Алгоритм унификации необходим для подбора знаний. Результат -- можно ли унифицировать два терма.
	\item \textbf{Какое первое состояние резольвенты?} -- В резольвенте находится вопрос.
	\item \textbf{Как меняется резольвента?} -- Преобразовании резольвенты выполняется с помощью операции <<редукция>>.
	\item \textbf{В каких пределах программы уникальны переменные?} -- Именованные -- в пределах предложения, анонимные -- уникальны всегда.
	\item \textbf{Как применяется подстановка, полученная с помощью алгоритма унификации?} -- Выполняется конкретизация переменной, если она несвязная.
	\item \textbf{В каких случаях запускается механизм отката?} -- Если решение не найдено, при этом резольвента непуста, и из данного состояния невозможно перейти в новое.
\end{enumerate}