\chapter{Контрольные вопросы}
\begin{enumerate}
	\item \textbf{В каком фрагменте программы сформулировано знание?} -- В заголовке правила.
	\item \textbf{Что содержит тело правила?} -- Условия истинности отношения.
	\item \textbf{Что дает использование переменных при формулировании знаний? В чем отличие формулировки знания с помощью термов с одинаковой арностью при использовании одной переменной и при использовании нескольких?} -- Переменные позволяют получить множество наборов значений, при которых высказывание становится истинным.
	\item \textbf{С каким квантором переменные входят в правило, в каких пределах переменная уникальна?} -- Переменные входят в правило с квантором существования. Именованная переменная уникальна в пределах одного предложения, анонимная -- всегда.
	\item \textbf{Какова семантика (смысл) предложений раздела DOMAINS? Когда, где и с какой целью используется это описание?} -- Раздел DOMAINS позволяет определить природу объекта.
	\item \textbf{Какова семантика (смысл) предложений раздела PREDICATES? Когда, где и с какой целью используется это описание?} -- Раздел PREDICATES позволяет с помощью использования составного терма обозначить структуру объекта (описать знание), обозначенную аргументами.
	\item \textbf{Унификация каких термов запускается на самом первом шаге работы системы? Каковы назначение и результат использования алгоритма унификации?} -- 
	В начале программы выполняется унификация двух термов: вопроса и первого в базе знаний. Унификация предназначена для подбора знаний таких, чтобы система смогла ответить на вопрос <<да>>. Её результатом является определение того, что могут ли быть унифицированы два терма или нет.
	\item \textbf{В каком случае запускается механизм отката?} -- Механизм отката запускается в случае ошибки, чтобы система смогла вернуться на предыдущий этап.
	
\end{enumerate}